\chapter{Zaključak i budući rad}
		
		 Zadatak naše grupe bio je razvoj web aplikacije pod nazivom "SatCom". Ideja aplikacije je da se korisnicima omogući komunikacija sa satelitima u mreži SatNogs. Nakon 12 tjedana timskog rada, ostvarili smo zadani cilj i projekt je završen. Projekt je imao tri faze.\newline
		 Prva faza je uključivala okupljanje tima, razgovor o generalnim idejama za aplikaciju, izražavanje pojedinačnih interesa i želja za radom u prvom ciklusu predaje, te konačno podjela zadataka za prvi ciklus. Formirala su se dva podtima. U prvom podtimu članovi su radili na implementaciji funkcionalnosti na frontendu. U drugom podtimu su članovi su radili na implementaciji funkcionalnosti na backendu. Rad na dokumentaciji podijeljen je na više manjih zadataka koje su pojedinci samostalno preuzimali. Prva je faza projekta trajala do kolokviranja prvog ciklusa projekta.\newline
		 U drugoj je fazi naglasak bio na implementaciji aplikacije, te se i ovdje zadržala podjela u dva podtima koji su radili na programskom ostvarenju aplikacije. U ovoj je fazi dovršena implementacija aplikacije, a trajala je do demonstracije alfa inačice aplikacije.\newline
		 Treća i konačna faza je uključivala izradu raznih UML dijagrama, ispitivanje sustava, pronalazak i ispravak grešaka. U ovoj fazi također je zadržana podjela na dva podtima te su zadatke vezane uz dokumentaciju članovi tima preuzimali samostalno. Ova je faza trajala do završetka projekta, prije konačne predaje i kolokviranja drugog ciklusa.\newline
		 Izgrađenu aplikaciju je moguće proširiti na mnogo načina. Jedna od ideja je da se implementira stranica koja bi prikazivala geolokacije pojedinih stanica i informacije o njima poput uspješnosti u slanju poruka.
		 Sudjelovanje u ovom projektu je bilo vrijedno iskustvo za sve članove tima. Svima nama je ovo bio prvi ozbiljniji grupni projekt, i snašli smo se jako dobro. Konflikata u timu nije bilo, a suradnja i komunikacija su bili na iznimno zadovoljavajućoj razini. Većini nas je ovaj projekt bio prvi ozbiljniji dodir s tehnologijama poput Gita i Latex. Naučili smo koristiti neke moderne radne okvire pri izradi web aplikacija. Iznimno smo zadovoljni postignutim rezultatima i timskim radom koji je do tih rezultata doveo.



		
		\eject 